\section{Revisão Teórica}
\subsection{Segmentação de Imagens}

Segmentação de imagens é um dos passos do processamento de imagens, e consiste em subdividir uma imagem em múltiplos segmentos de forma que cada segmento se apresente de forma diferente para o usuário [Image Segmentation Techniques: A Survey]. Ou, ainda, podemos definir segmentação de imagens como o processamento de imagem que subdivide uma imagem \textit{f(x,y)} em um subconjunto não vazio \textit{f1}, \textit{f2}, ..., \textit{fn} de imagens contínuas e desconexas que oferecem informações sobre a característica em que foi feita a segmentação [A SURVEY OF IMAGE SEGMENTATION TECHNIQUES].

Pelo fato de cada imagem ter características únicas, não há um algortimo universal para se fazer segmentação de imagem [Image Segmentation Techniques: A Survey], mas, de acordo com [LIVRO] os algoritmos de segmentação são baseados geralmente em duas estratégias: \textbf{discotinuidade} e \textbf{semelhança}. A primeira tenta explorar mudanças bruscas na imagem, como, por exemplo, em arestas, enquanto a segunda se baseia em dividir a imagem em regiões que apresentem alguma característica em comum, por exemplo, um valor mínimo (\textit{threshold}) em escala de cinza.