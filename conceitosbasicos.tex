\section{Revisão Teórica}
\paragraph*{}Segmentação de imagens é um dos passos do processamento de imagens, e consiste em subdividir uma imagem em múltiplos segmentos de forma que cada segmento se apresente de forma diferente para o usuário [Image Segmentation Techniques: A Survey]. Ou, ainda, podemos definir segmentação de imagens como o processamento de imagem que subdivide uma imagem \textit{f(x,y)} em um subconjunto não vazio \textit{f1}, \textit{f2}, ..., \textit{fn} de imagens contínuas e desconexas que oferecem informações sobre a característica em que foi feita a segmentação [A SURVEY OF IMAGE SEGMENTATION TECHNIQUES].

Pelo fato de cada imagem ter características únicas, não há um algortimo universal para se fazer segmentação de imagem [Image Segmentation Techniques: A Survey], mas, de acordo com [LIVRO] os algoritmos de segmentação são baseados geralmente em duas estratégias: \textbf{discotinuidade} e \textbf{semelhança}. A primeira tenta explorar mudanças bruscas na imagem, como, por exemplo, em arestas, enquanto a segunda se baseia em dividir a imagem em regiões que apresentem alguma característica em comum, por exemplo, um valor mínimo (\textit{threshold}) em escala de cinza.

Para um melhor entendimento de sgmentação de imagens, neste capitúlo serão discutidos alguns conceitos e métodos apontados na literatura que são usados no processo de segmentação.

\subsection{Detecção de Discontinuidades}
\paragraph*{}Segmentação por detecção de discontinuidade visa explorar as mundanças abruptas que existem em uma imagem em mudanças de um objeto para outro ou de um cenário para outro. Esta seção se focará em três tipos de discontinuidade: ponto, linha e aresta, todas aplicadas em imagens em tons de cinza.

\subsubsection{Detecção de Ponto}
\paragraph*{}A dtecção de ponto se baseia em explorar o quão diferente um ponto é de seus vizinhos. Para, isto se usa uma máscara que percorre a imagem, pasando por todos os pixels e compara seu valor com os pixel ao redor, se esta diferença for maior que um limite (\textit{threshold}) o ponto é um ponto de descontinuidade.

\begin{figure}[h]
\[ \left( \begin{array}{ccc}
-1 & -1 & -1 \\
-1 &  8 & -1 \\
-1 & -1 & -1 \end{array} \right)\] 
\caption{Exemplo de máscara para encontrar discontinuidade de ponto}
\end{figure}

\subsubsection{Detecção de Linha}
\paragraph*{}A detecção de linha consiste em procurar por linhas nas diversas direções em uma imagem. Para, isto se usa uma máscara que percorre a imagem. Como podemos ter linhas nas mais diversas orientações, cada linha tem uma máscara que otimiza sua detecção, como mostrado na \ref{fig:discontinuidadedelinha}.

\begin{figure}[h]
	\begin{minipage}{0cm}
		\[ \left( \begin{array}{ccc}
		-1 & -1 & -1 \\
		2 & 2 & 2 \\
		-1 & -1 & -1 \end{array} \right)\]
		\caption{horizontal}
	\end{minipage}
	\begin{minipage}{10cm}
		\[ \left( \begin{array}{ccc}
		-1 & 2 & -1 \\
		-1 & 2 & -1 \\
		-1 & 2 & -1 \end{array} \right)\]
		\caption{vertical}
	\end{minipage}
	\begin{minipage}{10cm}
		\[ \left( \begin{array}{ccc}
		2 & -1 & -1 \\
		-1 & 2 & -1 \\
		-1 & -1 & 2 \end{array} \right)\]
		\caption{$+45^{o}$}
	\end{minipage}
	\begin{minipage}{10cm}%
		\[ \left( \begin{array}{ccc}
		-1 & -1 & 2 \\
		-1 & 2 & -1 \\
		2 & -1 & -1 \end{array} \right)\]
		\caption{$-45^{o}$}
	\end{minipage}
\label{fig:discontinuidadedelinha}
\end{figure}
