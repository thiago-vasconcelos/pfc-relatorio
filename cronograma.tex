\section{Cronograma}
\FloatBarrier
\paragraph{}Por ser um trabalho mais voltado para a pesquisa foi criado um cronograma voltado majoritariamente para a parte teórica. 
\paragraph{}Primeiramente, o foco está no entendimento de conceitos básicos de sistemas de arquivos e de sistemas de arquivos distribuídos, pois há uma nomenclatura própria que permeia toda a teoria que vem em seguida que deve ser entendida assim como as técnicas básicas utilizadas em todos os sistemas de arquivos.
\paragraph{}Após este entendimento começa-se uma pesquisa voltada para dois sistemas de arquivos distribuídos específicos: o \textit{GoogleFS} e o \textit{Lustre}. Terminado o estudo em separado faz-se uma comparação teórica entre os dois e ao final é previsto uma análise de desempenho com o \textit{Lustre}.
\paragraph{}As avaliações previstas são as seguintes:
\begin{itemize}
\item Uma Verificação Especial em Outubro;
\item Uma Verificação Corrente em Março;
\item Uma Verificação Final em Junho.
\end{itemize}

