\section{Introdução}
\subsection{Motivação}
\paragraph{}O uso de plataformas como clusters tem se tornado cada dia mais popular. O cluster é uma solução de baixo custo que oferece alto desempenho para a resolução de problemas que exigem alto poder computacional. Um dos quesitos de um cluster é oferecer um sistema de arquivos único acessado por todos seus nós. Algumas aplicações de alto desempenho dependem de um sistema de arquivos eficiente para obter um bom desempenho. Sendo assim, é necessário um constante estudo e desenvolvimento de novas técnicas com o intúito de alcançar sistemas de arquivos distribuídos que oferecem bom desempenho.

\paragraph{}O foco é estudar as várias técnicas utilizadas por diversos sistemas de arquivos distribuídos realizando uma comparação dois deles existentes no mercado. Deseja-se, também, agregar conhecimento ao Instituto Militar de Engenharia para futuras aplicações na área de processamentos de alto desempenho.

\subsection{Objetivos}
\paragraph{}O objetivo do trabalho é estudar as técnicas utilizadas na implementação de sistemas de arquivos distribuídos, nos concentrando em dois sistemais mais utilizados no mercado: \textit{GoogleFS} e o \textit{Luster}. A partir deste estudo comparativo deseja-se avaliar as diferentes técnicas usadas pelos sistemas de arquivos tanto experimentalmente quanto teoricamente com o objetico de averiguar quais técnicas foram mais efetivas.

\subsection{Estrutura do trabalho}
\paragraph{}Primeiramente, haverá uma ambientação nos conceitos de sistemas de arquivos convencionais e distribuídos. Em seguida, será feito uma comparação teórica entre Sistema de Arquivos existentes, analisando o tratamento de cada sistema para cada técnica abordada.

\paragraph{}No segundo capítulo, são tratados os conceitos relevantes do sistema de arquivos, como a maneira na qual os arquivos são armazenados em disco e suas topologias. Também serão abordados, conceitos relativos à sistemas de arquivos distribuídos, tratando das arquiteturas existentes, dos sistemas de nomeação, das sincronizações de dados e das respectivas nuancias.

\paragraph{}No terceiro capítulo é apresentado um experimento onde são instalados dois sistemas de arquivos distribuídos em um cluster de computadores de baixo custo (\textit{Raspberry Pi}).

\paragraph{}Por fim o cronograma com as etapas planejadas até o final da pesquisa é apresentado no capítulo 4 e, em seguida, a conclusão da pesquisa está no capítulo 5.
